\chapter{T\'ITULO}
%
Sistema de informaci\'on para la gesti\'on de certificaciones en l\'inea.\\%
%
\chapter{PROBLEMA}
\section{DESCRIPCI\'ON DEL PROBLEMA}
%
El incremento de la comunidad estudiantil en la Escuela Colombiana de Carreras Industriales ha impactado de forma directa sobre la gesti\'on de los documentos estudiantiles, tales como las certificaciones.\\%
%
\\El proceso de las certificaciones deber\'ia tratarse de forma r\'apida, pero hoy en d\'ia, es una tarea que toma por lo menos ocho d\'ias h\'abiles y muchas veces, por incremento de otras tareas, este proceso ha sido detenido.\\%
%
\\Diariamente se observa filas bastante extensas, generando molest\'ias entre la comunidad estudiantil y opacando la imagen de esta \'area.%
%
\section{FORMULACI\'ON DEL PROBLEMA}
%
Se ha podido observar que los procesos que involucran la interactividad con los estudiantes, se vuelven m\'as eficientes en medida que las tareas se lleven a cabo v\'ia internet. Tareas como la inscripci\'on y cancelaci\'on de materias, entre otras, han demostrado que los medios interactivos en internet, descongestiona las instalaciones, adem\'as de facilitar los procesos para los estudiantes.\\%
%
\\?`Bajo qu\'e plataforma ser\'ia preciso desarrollar una herramienta que permita la implementaci\'on y gesti\'on de un sistema de informaci\'on que permita certificaciones en linea?.%