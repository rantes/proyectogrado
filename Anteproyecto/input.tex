% Copyright (C) 2000 Sergio Mendoza sergio@mrao.cam.ac.uk
% Astrophysics Group
% Cavendish Laboratory
% Cambridge UK
%
% This program is free software; you can redistribute it and/or modify
% it under the terms of the GNU General Public License as published by
% the Free Software Foundation; either version 2 of the License, or
% (at your option) any later version.
%
% This program is distributed in the hope that it will be useful,
% but WITHOUT ANY WARRANTY; without even the implied warranty of
% MERCHANTABILITY or FITNESS FOR A PARTICULAR PURPOSE. See the
% GNU General Public License for more details.
%
% You should have received a copy of the GNU General Public License
% along with this program; if not, write to the Free Software
% Foundation, Inc., 59 Temple Place, Suite 330, Boston, MA 02111-1307 USA

% Last modified: 21/July/2008 by Esteban Ricalde e_ricalde@yahoo.com.mx
\usepackage{fancyhdr}
\usepackage[a4paper]{geometry}
\usepackage[small,bf]{caption}
%\usepackage{mathtools}
%\usepackage{indentfirst}
\pagestyle{fancy}

% Coloca una línea en los encabezados
\fancyhf{}
\fancyhead[L,RO]{}
\fancyhead[LO]{}
\fancyhead[R]{}
%\fancyhead[CO,CE]{\includegraphics[width=1.00\textwidth]{images/header.png}}
% para mostrar el paginado
\fancyfoot[RO, R] {\thepage}
\renewcommand{\headrulewidth}{0cm}
% Cambia la estructura de una página en blanco
\fancypagestyle{plain}{
%	\fancyhead[CO,CE]{\includegraphics[width=1.00\textwidth]{images/header.png}}
	\renewcommand{\headrulewidth}{0pt}
}

% Cambia el ancho del encabezado
\setlength{\headwidth}{16.5cm}

% Cambia el espacio para el encabezado
\setlength{\voffset}{10pt}
\setlength{\headheight}{74.07pt}
\setlength{\headsep}{5pt}

% Cambia el margen de los pies de figura
\setlength{\captionmargin}{10pt}

% Borra la palabra Capítulo del \chaptermark:
\renewcommand{\chaptermark}[1]{\markboth{\MakeUppercase{\thechapter. \dotfill #1}}{}}
\renewcommand{\thechapter}{\arabic{chapter}}
% Quita la palabra capitulo
\addto\captionsspanish{\renewcommand{\chaptername}{}}

% Define comando para colocar páginas en blanco antes de iniciar cada capítulo
%\newcommand{\clearemptydoublepage}{\newpage{\pagestyle{empty}
%\cleardoublepage}}
%\newcommand{\HRule}{\rule{\linewidth}{0.5mm}}	


\usepackage{titlesec}
%cambiar los capitulos
\titleclass{\chapter}{straight}
\titleformat{\chapter}[block]
	{\setlength{\parindent}{0pt} \raggedright\normalfont\normalfont\bfseries}{\thechapter.}{5pt}{\MakeUppercase}{}
\titlespacing*{\chapter}{12pt}{12pt}{12pt}
% cambiar las secciones
\titleformat{\section}[block]
	{\setlength{\parindent}{0pt} \raggedright\normalfont\normalfont\bfseries}{\thesection.}{5pt}{\MakeUppercase}{}
\titlespacing*{\section}{12pt}{12pt}{12pt}
% definir la fuente
\renewcommand*\rmdefault{Arial}